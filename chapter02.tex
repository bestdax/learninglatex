\chapter{\LaTeX 入门}

\section{hello world}
按惯例,我们首先在文本编辑器中写一个hello world,让\LaTeX 先用起来。如果是完全新手的话,建议首先使用发行版自带的编辑器,或者是overleaf的在线编辑器,因为对运行环境的集成度高,所以容易上手。

\begin{texlst}[listing only, label={code:hello}, nameref={Hello World示例代码}]
\documentclass{article}
\begin{document}
hello world
\end{document}
\end{texlst}

将上面的代码输入到编辑器中,然后编译(在发行版的编辑器会有编译的按钮),完成之后就会看到生成的PDF文件中显示的“hello
world”。

下面是对于这段代码的说明:
\begin{enumerate}
	\forcsvlist{\item}{
	      第一个语句\texinline{\documentclass{article}}是指定文档类型,这里指定的是article类型,
	      \texinline{\begin{document}}和\texinline{\end{document}}之间是正文区,
	      文档类型与正文之间称为导言区,这里可以加载其他宏包或者进行一些设定,这个示例里面这些都没有,我们会在后面看到
	      }
\end{enumerate}

\section{文档类型}
文档类型是\LaTeX 基于\TeX
引擎做的一层抽象,使得文档内容与格式能够分离,这样我们写文档的时候可以更加专注于内容。但是实际上除非是有一个完全合乎规范的文档模板,写作者本身还是需要具备一定的\LaTeX
知识才能顺利地写作的。

\LaTeX 内置了数种文档类型,其中常用的:article、report和book。各个文档类型的适用范围如下表:
\noindent
\begin{table}[ht]
	\begin{tabular}{lll} \toprule
		文档类型 & 说明     & 支持的章节标号                                    \\ \midrule
		book     & 长篇书籍 & part, chapter, section, subsection, subsubsection \\
		report   & 中篇报告 & chapter, section, subsection, subsubsection       \\
		article  & 短篇文章 & section, subsection, subsubsection                \\
		\bottomrule
	\end{tabular}
	\caption{\LaTeX 内置常用文档类型}
\end{table}

回到``hello world''的示例,如果把“hello
world”改成“你好,世界”,你将看不到相应的输出。这是因为article文档类型默认的字体并不支持中文,我们可以通过设置正确的字体,或者选择一个支持中文的文档类型。这里我们简单介绍一下ctex宏包,它提供了四种文档类型:ctexart、ctexrep、ctexbook及ctexbeamer,前面三个是基于\LaTeX
的基础文档类型做的汉化版本,ctexbeamer是用来制作幻灯片的。

\AmS (美国数学学会)也提供了一套文档类型,分别为amsart和 amsbook,对数学公式的排版提供了更多的支持。当然还有很多很多的文档类型,可以根据自身的需要去使用。

\section{\LaTeX 的三种语句} \label{三种语句}
\LaTeX 的语句分三种,命令、文本和注释。如\enquote{hello word}示例中的
\texinline{\documentclass{article}}就是一条命令语句,以反斜杠\textbackslash 开始,
document-class为命令名,只能用英文字母。这个规定涉及到\TeX 中的字符分类码(code
category),但是分类码也是可以修改的,这个以后再讲。documentclass后面的花括号内是命令的参数。命令一般还会有选项,比如:
\\	\texinline{\documentclass[a4paper]{article}},用来设置纸张大小为A4,至于其他的选项后面再介绍。

\nameref{code:hello}中的\enquote{hello
	world}就是文本。注释目前还没有遇到,注释的作用是向文档中添加说明或解释,这些内容在编译时不会被包含在最终的PDF或其他格式的输出文件中。注释对于理解代码、临时移除代码段或测试文档中的特定部分非常有用。注释用\%开始。
\begin{texlst}[label={code:comments}]
	这是可见文本

	% 这是注释不可见
\end{texlst}

\section{\LaTeX 中的特殊字符}
在第\ref{三种语句}节中,讲到注释是以\%开始的,那么很自然的一个疑问,\%号本身该如何输出呢?这就引出了我们本节的内容特殊字符。特殊字符分为两类,一类是键盘上有字符,这种特殊字符由于用作控制字符,所以不能直接输入。第二种就是键盘上本来就没有的符号,比如\alpha\beta
等等。
\begin{table}[ht]
	\centering
	\begin{tabular}{lll} \toprule
		符号      & 作用                           & 输出方式                   \\ \midrule
		\verb|#|  & 宏参数定义                     & \texinline{\#}             \\
		\verb|$|  & 开启数学环境                   & \texinline{\$}             \\
		\verb|^|  & 用于数学环境中表示上标         & \texinline{\^{}}           \\
		\verb|_|  & 用于数学环境中表示下标         & \texinline{\_}             \\
		\verb|&|  & 用于一些环境中表示对齐         & \texinline{\&}             \\
		\verb|{}| & 成对出现,表示范围,或者作用域 & \texinline{\{ \}}          \\
		\verb|~|  & 表示此处不能断开               & \texinline{\~{}}           \\
		\verb|\|  & 开始一个命令                   & \texinline{\textbackslash} \\ \bottomrule
	\end{tabular}
	\caption{控制字符}
\end{table}

\pagebreak
第二类特殊符号数量众多,可以在命令行输入以下命令:
\begin{shellcmd}
	texdoc symbols
\end{shellcmd}

里面有各种各样的符号的输入方法。

上面的输出方式中,有两个在后面加上了\texinline|{}|,这种写法是因为这些命令会将直接将紧随命令之后或者紧随命令空格后的一个字符当做参数,\texinline|{}|表示一个内容为空的范围,不会有任何字符成为参数,也就可以输出符号本身。\texinline|\^ a|和\texinline|\^a|都会输出\^a。

\section{空格与换行}
\subsection{连续空格}
在\LaTeX 中,空格与换行的处理与日常使用的Word之类的软件中的方式有很大的不同。首先,在\LaTeX
中连续的空格被视为一个空格,这样可以做到规范文本格式。比如:
\begin{texlst}
	这是 关于空格    的示例。
\end{texlst}

\subsection{空格大小}
\LaTeX
中的空格的长度与字体的大小相关。在印刷术语中,把一个大写字母M的宽度称为1em,一个小写字母x的高度称为1ex,1em相当于字体的大小。比如\LaTeX
正文的默认大小为10pt,也就是1em约等于10pt。另外,1en等于1em的一半(n是半个m)。在\LaTeX
中宽空格等于1em,可以用\texinline{\quad}来输出。另外,在数学环境下,用mu(math unit)这个单位,$1\text{em}=18\text{mu}$。在\LaTeX
中空白分别可分行与不可分行两种,区别就在于它们所在的地方能不能被分割两行。

下表是\LaTeX 中提供的空格的命令以及相应的大小。

\begin{table}[htpb]
	\centering
	\begin{tabular}{lll} \toprule
		空格命令                  & 长度       & 备注         \\ \midrule
		\verb|\,|                 & $1/6$\,em  & 空\,格       \\
		\verb|\>|                 & $2/9$\,em  & 空\>格       \\
		\verb|\;|                 & $5/18$\,em & 空\;格       \\
		\verb|\|\textvisiblespace & $1/3$\,em  & 空\ 格       \\
		\verb|\enskip|            & $0.5$\,em  & 空\enskip 格 \\
		\verb|\!|                 & $-1/6$\,em & 空\!格       \\
		\verb|\quad|              & $1$\,em    & 空\quad 格   \\
		\verb|\qquad|             & $2$\,em    & 空\qquad 格  \\
		\verb|\hskip{长度值}|     & 任意长度   &              \\ \bottomrule
	\end{tabular}
	\caption{可分行空白}
\end{table}

\begin{table}[htpb]
	\centering
	\begin{tabular}{lll} \toprule
		空格命令               & 长度       & 备注             \\ \midrule
		\verb|\thinspace|      & $1/6$\,em  & 空\thinspace 格  \\
		\verb|\medspace|       & $2/9$\,em  & 空\medspace 格   \\
		\verb|\thickspace|     & $5/18$\,em & 空\thickspace 格 \\
		\verb|~|               & $1/3$\,em  & 空~格            \\
		\verb|\enspace|        & $0.5$\,em  & 空\enspace 格    \\
		\verb|\negthinspace|   & $-1/6$\,em & 空\!格           \\
		\verb|\hspace{长度值}| & 任意长度   &                  \\ \bottomrule
	\end{tabular}
	\caption{不可分行空白}
\end{table}

这里有一个没有明说的命名规则,大家可以看到以skip命名的命令可以分行,而space的命令则不可以。另外,虽然\texinline{\quad}和\texinline{\qquad}没有对应的不可分行的版本,但是可以用\texinline{\hspace{1em}}或者\texinline{\hspace{2em}}来实现。

\subsection{换行和分段}
在\LaTeX 中,段落中间的空行(多个空行视为一个)会开始新的段落。

%\begin{noindent}
\begin{texlst}
这是
一个句子,
中间的回车被忽略。

开始新的一段。


多个空行被视为一个。
\end{texlst}
%\end{noindent}

如果在行文中需要强制换行,可以使用\texinline{\\}。另外,\texinline{\\}命令还可以接受参数来控制纵向的距离。

\begin{texlst}
	这里\\换行

	然后\\[2em]空两行
\end{texlst}

\texinline{\\*}命令也是强制换行,但是防止在此处分页。\texinline{\newline}的效果与\texinline{\\}一样,因为\texinline{\\}在输入方面的优势,一般都用\texinline{\\}。另外\texinline{\linebreak}也可以起到强制换行的效果,但是\texinline{\linebreak}会影响到分页,\texinline{\newline}则不会。

此外,\LaTeX 中用\texinline{\par}来显式的起始一个段落。

\begin{texlst}
	段落1\par 段落2
\end{texlst}
