\chapter{\LaTeX 入门}

\section{Hello World}
按惯例,我们首先在文本编辑器中写一个hello world,让\LaTeX 先用起来。如果是完全新手的话,建议首先使用发行版自带的编辑器,或者是overleaf的在线编辑器,因为对运行环境的集成度高,所以容易上手。

\begin{texlst}[listing only, label={code:hello}, nameref={Hello World示例代码}]
\documentclass{article}
\begin{document}
Hello World
\end{document}
\end{texlst}

将上面的代码输入到编辑器中,然后编译(在发行版的编辑器会有编译的按钮),完成之后就会看到生成的PDF文件中显示的“Hello
World”。

下面是对于这段代码的说明:
\begin{enumerate}
	\forcsvlist{\item}{
	      第一个语句\texinline{\documentclass{article}}是指定文档类型,这里指定的是\texinline{artile}类型。,
	      \texinline{\begin{document}}和\texinline{\end{document}}之间是正文区。,
文档类型与正文之间称为导言区,这里可以加载其他宏包或者进行一些设定,这个示例里面这些都没有,我们会在后面看到。
}
\end{enumerate}

\section{文档类型}\label{sec:doctype}
文档类型是\LaTeX 基于\TeX
引擎做的一层抽象,使得文档内容与格式能够分离,这样我们写文档的时候可以更加专注于内容。但是实际上除非是有一个完全合乎规范的文档模板,写作者本身还是需要具备一定的\LaTeX
知识才能顺利地写作的。

\LaTeX 内置了数种文档类型,其中常用的:\texinline{article}、\texinline{report}和\texinline{book}。各个文档类型的适用范围如下表:
\noindent
\begin{table}[ht]
	\centering
	\begin{tabular}{lll} \toprule
		文档类型 & 说明     & 支持的章节标号                                    \\ \midrule
		book     & 长篇书籍 & part, chapter, section, subsection, subsubsection \\
		report   & 中篇报告 & chapter, section, subsection, subsubsection       \\
		article  & 短篇文章 & section, subsection, subsubsection                \\
		\bottomrule
	\end{tabular}
	\caption{\LaTeX 内置常用文档类型}
\end{table}

回到``Hello World''的示例,如果把“Hello
World”改成“你好,世界”,你将看不到相应的输出。这是因为\texinline{article}文档类型默认的字体并不支持中文,我们可以通过设置正确的字体,或者选择一个支持中文的文档类型。这里我们简单介绍一下\texinline{ctex}宏包,它提供了四种文档类型:\texinline{ctexart}、\texinline{ctexrep}、\texinline{ctexbook}及\texinline{ctexbeamer},前面三个是基于\LaTeX
的基础文档类型做的汉化版本,ctexbeamer则是用来制作幻灯片的。

美国数学学会 (\AmS)也提供了一套文档类型,分别为\texinline{amsart}和 \texinline{amsbook},对数学公式的排版提供了更多的支持。当然还有很多很多的文档类型,可以根据自身的需要去使用。

\section{文档结构}
在\nameref{sec:doctype}一节中,我们了解了文档有章节之类的结构,除此之外,文档还有标题,作者,日期以及目录这些信息。
标题、作者及日期的申明放在导言区,命令分别是\texinline{\title}、\texinline{\author}、\texinline{\date}。申明了这些信息后不会直接显示,需要在正文区用\texinline{\maketitle}命令来显示这些信息。

注意,如果有多位作者,名字之间用\texinline{\and}命令来连接。另外,日期可以用\texinline{\today}来生成当天的日期。

目录的生成也很简单,用\texinline{\tableofcontents}命令就可以自动生成。如果某些章节你不希望出现在目录里,可以用相应的章节命令的星号版本。

下面是一个简单的示例:
\begin{texlst}[listing only]
	%!tex program = lualatex
	\documentclass{ctexart}
	\title{我的巨著}
	\author{大象同学 \and 我自己}
	\date{\today}
	\begin{document}
	\maketitle
	\tableofcontents

	\section{我有一个伟大的想法}
	balabala
	\section*{这个不会出现在目录里面}
	这个小节标题不会有编号,只有标题。

	\end{document}
\end{texlst}

\clearpage

\section{页面布局}
\drawpage
上面的布局图是用\texinline{layouts}宏包的\texinline{\drawpage}命令绘制的。与我们的直观理解类似,一个页面分为页眉、页脚、边注区及正文区,每个区域的尺寸及格式都是可以单独调整与设置的,这些内容我们后面再深究。

\section{\LaTeX 的三种语句} \label{sec:text type}
\LaTeX 的语句分三种,命令、文本和注释。
我们来看一个命令的示例:
\begin{texlst}[listing and text]
	\framebox[3cm]{这是一个盒子}
\end{texlst}

在 \LaTeX{} 中,命令通常用于控制文档的格式和内容。 \LaTeX{} 命令主要有两种格式:

第一种是以反斜杠 (\textbackslash) 开头,后面跟随一个或多个字母的命令。例如,\texinline{\begin}、\texinline{\end}、\texinline{\section} 等。这些命令通常用于结构化文档、插入标题、设置格式等。比如,\texinline{\section{引言}} 将插入一个名为“引言”的节标题。

第二种是以反斜杠 (\texinline{\}) 开头,后面跟随一个非字母字符的命令。例如,\texinline{\\} 用于插入换行符,\texinline{\#}
用于插入井号 (\#),\texinline{\^} 用于数学环境中输出上标。这些命令通常用于插入特殊字符或执行特定的排版操作。

\nameref{code:hello}中的\enquote{hello
	world}就是文本。注释目前还没有遇到,注释的作用是向文档中添加说明或解释,这些内容在编译时不会被包含在最终的PDF或其他格式的输出文件中。注释对于理解代码、临时移除代码段或测试文档中的特定部分非常有用。注释用\%开始。
\begin{texlst}[label={code:comments}]
	这是可见文本

	% 这是注释不可见
\end{texlst}

\section{\LaTeX 中的特殊字符}
在\nameref{sec:text type}中,讲到注释是以\%开始的,那么很自然的一个疑问,\%号本身该如何输出呢?这就引出了我们本节的内容特殊字符。特殊字符分为两类,一类是键盘上有字符,这种特殊字符由于用作控制字符,所以不能直接输入。第二种就是键盘上本来就没有的符号,比如\alpha\beta
等等。
\pagebreak
\begin{table}[ht]
	\centering
	\begin{tabular}{lll} \toprule
		符号      & 作用                           & 输出方式                   \\ \midrule
		\verb|#|  & 宏参数定义                     & \texinline{\#}             \\
		\verb|$|  & 开启数学环境                   & \texinline{\$}             \\
		\verb|^|  & 用于数学环境中表示上标         & \texinline{\^{}}           \\
		\verb|_|  & 用于数学环境中表示下标         & \texinline{\_}             \\
		\verb|&|  & 用于一些环境中表示对齐         & \texinline{\&}             \\
		\verb|{}| & 成对出现,表示范围,或者作用域 & \texinline{\{ \}}          \\
		\verb|~|  & 表示此处不能断开               & \texinline{\~{}}           \\
		\verb|\|  & 开始一个命令                   & \texinline{\textbackslash} \\ \bottomrule
	\end{tabular}
	\caption{控制字符}
\end{table}

第二类特殊符号数量众多,可以在命令行输入以下命令:
\begin{shellcmd}
	texdoc symbols
\end{shellcmd}

里面有各种各样的符号的输入方法。

上面的输出方式中,有两个在后面加上了\texinline|{}|,这种写法是因为这些命令会将直接将紧随命令之后或者紧随命令空格后的一个字符当做参数,\texinline|{}|表示一个内容为空的范围,不会有任何字符成为参数,也就可以输出符号本身。\texinline|\^ a|和\texinline|\^a|都会输出\^a。

\section{空格}
在 LaTeX 中,空格不仅仅是文本中的空白区域;它们是文档格式化的重要组成部分,用于控制单词间的间隔、段落内的行距,甚至是数学公式中的符号间隔。理解 LaTeX 如何处理空格,以及如何控制空格的大小和位置,是制作高质量文档的关键。
\subsection{自动空格处理}
\LaTeX 通常自动管理文本中的空格。例如,无论你在源文件中输入多少空格,\LaTeX 通常只会显示一个标准的单词间隔。同样,连续的空格会被合并成一个,多余的空格会被忽略。这意味着你不必担心源文件中的空格数量,\LaTeX 会自动调整以保持良好的阅读体验。
\begin{texlst}
	这是 关于空格    的示例。
\end{texlst}

\subsection{空格大小}
\LaTeX
中的空格的长度与字体的大小相关。在印刷术语中,把一个大写字母M的宽度称为1em,一个小写字母x的高度称为1ex,1em相当于字体的大小。比如\LaTeX
正文的默认大小为10pt,也就是1em约等于10pt。另外,1en等于1em的一半(n是半个m)。在\LaTeX
中宽空格等于1em,可以用\texinline{\quad}来输出。另外,在数学环境下,用mu(math unit)这个单位,$1\text{em}=18\text{mu}$。在\LaTeX
中空白分别可分行与不可分行两种,区别就在于它们所在的地方能不能被分割两行。

下表是\LaTeX 中提供的空格的命令以及相应的大小。

\begin{table}[htpb]
	\centering
	\begin{tabular}{lrl} \toprule
		空格命令                  & 长度          & 备注                          \\ \midrule
		\verb|\,|                 & $3\text{mu}$  & \Rightarrow\,\Leftarrow       \\
		\verb|\>或\:|             & $4\text{mu}$  & \Rightarrow\>\Leftarrow       \\
		\verb|\;|                 & $5\text{mu}$  & \Rightarrow\;\Leftarrow       \\
		\verb|\|\textvisiblespace & $6\text{mu}$  & \Rightarrow\ \Leftarrow       \\
		\verb|\enskip|            & $9\text{mu}$  & \Rightarrow\enskip \Leftarrow \\
		\verb|\!|                 & $-3\text{mu}$ & \Rightarrow\!\Leftarrow       \\
		\verb|\quad|              & $18\text{mu}$ & \Rightarrow\quad \Leftarrow   \\
		\verb|\qquad|             & $36\text{mu}$ & \Rightarrow\qquad \Leftarrow  \\ \bottomrule
	\end{tabular}
	\caption{可分行空白}
\end{table}
\pagebreak

\begin{table}[htpb]
	\centering
	\begin{tabular}{lrl} \toprule
		空格命令              & 长度          & 备注                                 \\ \midrule
		\verb|\thinspace|     & $3\text{mu}$  & \Rightarrow\thinspace \Leftarrow     \\
		\verb|\medspace|      & $4\text{mu}$  & \Rightarrow\medspace \Leftarrow      \\
		\verb|\thickspace|    & $5\text{mu}$  & \Rightarrow\thickspace \Leftarrow    \\
		\verb|~|              & $6\text{mu}$  & \Rightarrow~\Leftarrow               \\
		\verb|\enspace|       & $9\text{mu}$  & \Rightarrow\enspace \Leftarrow       \\
		\verb|\negthinspace|  & $-3\text{mu}$ & \Rightarrow\negthinspace \Leftarrow  \\
		\verb|\negmedspace|   & $-4\text{mu}$ & \Rightarrow\negmedspace \Leftarrow   \\
		\verb|\negthickspace| & $-5\text{mu}$ & \Rightarrow\negthickspace \Leftarrow \\
		\bottomrule
	\end{tabular}
	\caption{不可分行空白}
\end{table}

你可能会觉得\texinline[showspaces]{\}\textvisiblespace 这个命令是多此一举,因为用空格就直接可以输出一个空格,这个命令还有存在的必要吗?我们来看下面一个示例:
\begin{texlst}[listing and text]
	\LaTeX is a typesetting system.

	\LaTeX\ is a typesetting system.
\end{texlst}

可以看到第一个写法\LaTeX
与is中间的空格消失了,因为空格或者其他非字母的字符是命令的界定符(除了几个特殊的\TeX
命令),在一个命令之后的空格就不会再输出空格。所以需要用\texinline[showspaces]{\}\textvisiblespace 来解决这个问题。

\subsection{数学模式中的空格}
在数学模式下,LaTeX
会自动调整符号和表达式之间的空格,以反映数学运算的语义。例如,函数名和变量之间会自动添加适当的空间。然而,你仍然可以使用上述命令来微调数学公式的布局。
\begin{texlst}[listing and text]
$a^2+b^2=c^2$
\end{texlst}

\subsection{避免自动空格处理}
有时你可能不希望 \LaTeX 自动合并或忽略空格,例如在代码示例或诗歌中。在这种情况下,可以使用 \texinline{verbatim} 环境,它会原样保留所有的空格和换行。

\subsection{小结}
\LaTeX 的空格处理机制提供了灵活性和控制力,使你能够在保持文档整洁的同时,对文本布局进行精确调整。通过理解 \LaTeX
中空格的处理机制,你可以创造出既美观又专业,同时符合特定需求的文档。

\section{换行和分段}
在 \LaTeX
编辑过程中,掌握如何控制文本的换行和分段是非常重要的技能,它直接影响到文档的可读性和美观度。不同于大多数文字处理器,\LaTeX 通常自动处理文本的布局,但有时你可能需要手动干预。
\subsection{自动换行}
\LaTeX 默认会自动进行换行和段落对齐。只要你在输入文本时按回车键结束一行并开始新的一行,\LaTeX
会将这些视为逻辑上的新行,但在输出的文档中并\emph{不会形成可见的换行符}。LaTeX 会根据页面宽度和段落内容自动进行换行。

\subsection{手动换行}
如果你需要在特定位置强制换行,可以使用双反斜杠 \texinline{\\} 或者 \texinline{\newline} 命令。
\texinline{\\*}命令也是强制换行,但是防止在此处分页。\texinline{\newline}的效果与\texinline{\\}一样,因为\texinline{\\}在输入方面的优势,一般都用\texinline{\\}。另外\texinline{\linebreak}也可以起到强制换行的效果,但是\texinline{\linebreak}会影响到分页,\texinline{\newline}则不会。
另外,\texinline{\\}命令还可以接受参数来控制纵向的距离。
\begin{texlst}[listing and text]
	这里\\换行

	然后\\[2em]空两行
\end{texlst}

但是请注意,过多的手动换行可能会破坏 LaTeX 的自然段落布局,导致不理想的间距问题。
\subsection{分段}
在 \LaTeX 中,创建一个新的段落非常简单,只需在段落之间插入一个空白行(多个空白行视为一个)即可。
\begin{texlst}[listing and text]
	这是
	一个句子,
	中间的回车被忽略。

	开始新的一段。



	多个空行被视为一个。
\end{texlst}

此外,\LaTeX 中用\texinline{\par}来显式的起始一个段落。

\begin{texlst}
	段落1\par 段落2
\end{texlst}

\subsection{小结}
在 \LaTeX 中,换行和分段是文本布局的基本组成部分。了解如何使用 \LaTeX 的自动换行功能以及如何在必要时进行手动干预,将有助于你创建结构清晰、格式一致的高质量文档。

\section{列表}
列表
是一种组织信息的有力工具,在学术论文、技术手册、报告、教程和几乎任何类型的文档中都非常常见。它们帮助读者快速浏览关键点,清晰地展示步骤或选项,以及区分不同概念或组成部分。在
\LaTeX 中,创建列表不仅容易,而且高度可定制,允许你根据文档的具体需求来调整样式和格式。
\subsection{无序列表(项目符号列表)}
在 LaTeX 中创建无序列表(也称为项目符号列表)非常简单,使用 \texinline{itemize} 环境即可。每个列表项都以 \texinline{\item} 命令开始。
\begin{texlst}[listing and text]
	\begin{itemize}
		\item 第一项
		\item 第二项
		\item 第三项
	\end{itemize}
\end{texlst}

\subsection{有序列表(编号列表)}
如果你需要一个有序列表,即带有数字或其他类型的编号的列表,那么应该使用 \texinline{enumerate} 环境。
\begin{texlst}[listing and text]
	\begin{enumerate}
		\item 第一项
		\item 第二项
		\item 第三项
	\end{enumerate}
\end{texlst}

\subsection{描述列表}
描述列表(或定义列表)用于提供术语及其定义。在 \LaTeX 中,使用 \texinline{description} 环境。
\begin{texlst}[listing and text]
	\begin{description}
		\item[术语一] 定义一
		\item[术语二] 定义二
		\item[术语三] 定义三
	\end{description}
\end{texlst}

\subsection{小结}
列表是 \LaTeX 文档中非常实用的功能,通过使用 \texinline{itemize}、\texinline{enumerate} 和 \texinline{description}
环境,你可以轻松地创建各种风格的列表,以适应你的文档需求。自定义项目符号和编号格式则让你能够进一步调整列表的外观,使其与文档的整体风格相匹配。自定义的部分我们放在后面讲。
