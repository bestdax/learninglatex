\chapter{\LaTeX 入门}

\section{hello world}
按惯例,我们首先在文本编辑器中写一个hello world,让\LaTeX 先用起来。如果是完全新手的话,建议首先使用发行版自带的编辑器,或者是overleaf的在线编辑器,因为对运行环境的集成度高,所以容易上手。

\begin{texlst}[listing only]
\documentclass{article}
\begin{document}
hello world
\end{document}
\end{texlst}
将上面的代码输入到编辑器中,然后编译(在发行版的编辑器会有编译的按钮),完成之后就会看到生成的PDF文件中显示的“hello
world”。

下面是对于这段代码的说明:
\begin{enumerate}
	\forcsvlist{\item}{
	      第一个语句\texinline{\documentclass{article}}是指定文档类型,这里指定的是article类型,
	      \texinline{\begin{document}}和\texinline{\end{document}}之间是正文区,
	      文档类型与正文之间称为导言区,这里可以加载其他宏包或者进行一些设定,这个示例里面这些都没有,我们会在后面看到
	      }
\end{enumerate}

\section{文档类型}
文档类型是\LaTeX 基于\TeX
引擎做的一层抽象,使得文档内容与格式能够分离,这样我们写文档的时候可以更加专注于内容。但是实际上除非是有一个完全合乎规范的文档模板,写作者本身还是需要具备一定的\LaTeX
知识才能顺利地写作的。

\LaTeX 内置了数种文档类型,其中常用的:article、report和book。各个文档类型的适用范围如下表:
\noindent
\begin{table}[ht]
	\begin{tabular}{lll} \toprule
		文档类型    & 说明   & 支持的章节标号                                           \\ \midrule
		book    & 长篇书籍 & part, chapter, section, subsection, subsubsection \\
		report  & 中篇报告 & chapter, section, subsection, subsubsection       \\
		article & 短篇文章 & section, subsection, subsubsection                \\
		\bottomrule
	\end{tabular}
	\caption{\LaTeX 内置常用文档类型}
\end{table}

回到``hello world''的示例,如果把“hello
world”改成“你好,世界”,你将看不到相应的输出。这是因为article文档类型默认的字体并不支持中文,我们可以通过设置正确的字体,或者选择一个支持中文的文档类型。这里我们简单介绍一下ctex宏包,它提供了四种文档类型:ctexart、ctexrep、ctexbook及ctexbeamer,前面三个是基于\LaTeX
的基础文档类型做的汉化版本,ctexbeamer是用来制作幻灯片的。

\AmS (美国数学学会)也提供了一套文档类型,分别为amsart和 amsbook,对数学公式的排版提供了更多的支持。当然还有很多很多的文档类型,可以根据自身的需要去使用。

\section{\LaTeX 的三种语句}
\LaTeX 的语句分三种,命令、文本和注释。如``hello word''示例中的
\texinline{\documentclass{article}}就是一条命令语句,以反斜杠\textbackslash 开始,
document-class为命令名,只能用英文字母。这个规定涉及到\TeX 中的字符分类码(code
category),但是分类码也是可以修改的,这个以后再讲。documentclass后面的花括号内是命令的参数。命令一般还会有选项,比如:
\\	\texinline{\documentclass[a4paper]{article}},用来设置纸张大小为A4,至于其他的选项后面再介绍。

``hello world''示例中的'
