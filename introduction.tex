\chapter{\LaTeX  简介}
\section{历史演变}
说起\LaTeX 我们就不得不从\TeX 说起。

\TeX 是由高德纳教授\footnote{Donald E. Knuth,
	斯坦福大学计算机教授,1992年退休}于1978年开发的排版系统,最初目的是为了排版他的著作《计算机程序设计艺术》。高教授发现当时存在的排版技术无法满足他对数学公式和高质量排版的需求,因此决定自己开发一个系统。另外,纯\TeX
只包含最低层的原始命令(primitives),为了增强易用性,高教授在纯\TeX 的基础上增加了一些宏集,开发出了所谓的plain \TeX
。即便这样,
plain \TeX 的学习门槛仍然相对较高。

plain \TeX 以其卓越的排版质量和对数学公式处理的能力而闻名,但同时因其复杂的宏包和学习曲线而让一些用户望而却步。

为了解决这个问题,Leslie Lamport在1980年代初期引入了\LaTeX ,这是一种基于\TeX 的宏包集合,它提供了一套更易于使用的高级宏命令,使得用户无需深入了解底层的\TeX 命令即可创建结构化和格式化的文档。\LaTeX 通过简化命令和提供文档类文件,极大地方便了学术论文、技术文档和书籍的排版工作。

\section{\LaTeX 的发展}
\begin{itemize}
	\item aa

\end{itemize}
\LaTeX 的发展经历了多个版本,其中\LaTeXe 是目前最广泛使用的版本,由Frank Mittelbach领导的\LaTeX 3项目团队在1994年发布。\LaTeXe 不仅改进了排版功能,还引入了对复杂文档结构的更好支持,如部分文档、章节和参考文献管理。

随着时间的推移,\LaTeX 社区不断发展壮大,出现了许多附加的宏包和工具,如BibTeX用于参考文献管理,以及各种用于数学公式、图形插入和定制排版的宏包。\LaTeX 的强大功能和灵活性使其成为学术界和出版界首选的排版工具之一。

\LaTeX 的另一个重要特点是其跨平台的特性,几乎可以在所有操作系统上运行,包括Windows、macOS和Linux。此外,\LaTeX 的源文件是纯文本格式,这使得版本控制和文档共享变得非常方便。

总而言之,\LaTeX 是一个功能强大、高度可定制的排版系统,它从最初的\TeX 发展而来,已经成为科技文档和学术出版物中不可或缺的工具。


