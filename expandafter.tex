\section{展开的顺序及规则}
在\LaTeX 中,控制序列展开是一个很令人头疼的事情。说起来比较抽象,我们用一个示例来演示一下。
\begin{texlst}[listing and text]
  \newcommand\cmda{abc}
  \uppercase{\cmda}
\end{texlst}

代码并没有如预期的那样输出ABC,在这个过程,先展开\texinline{\uppercase},然后展开\texinline|{}|,但是并不会展开\texinline|{}|中的需要进一步展开的内容,所以\texinline{\uppercase}没有获得任何参数。最后展开了\texinline{\cmda},所以就得到了我们看到的\texinline{abc}。

这里就到了\texinline{\expandafter}命令出场的时候了,我们可以让\texinline|{}|先不展开,先展开\texinline{\cmda},请参看下例:
\begin{texlst}[listing and text]
  \newcommand\cmda{abc}
  \uppercase\expandafter{\cmda}
\end{texlst}

这么看来似乎展开也不麻烦,但是这个只是最简单的情况,下面我们来看,如果有两个控制序列需要优先展开的情况。
\begin{texlst}[listing and text]
\newcommand\cmda{abc}
\newcommand\cmdb{def}
\uppercase\expandafter{\cmda\cmdb}
% 只会展开\cmda而不会展开\cmdb
\let\ea\expandafter

\uppercase\ea\ea\ea{\ea\cmda\cmdb}
\end{texlst}

来解释一下这个运行的过程是怎么样发生的。第一个\texinline{\ea}命令将第二个\texinline{\ea}命令保留下来,第三个\texinline{\ea}命令让\texinline|{}|不要展开,第四个\texinline{\ea}命令让\texinline{\cmda}不要展开,然后展开\texinline{\cmdb}。所以第一轮延后展开的结果变成了:\texinline{\ea{\cmda def}},这个就跟单个控制序列优先展开是一样的了。

如果要展开的控制序列数量进一步增加的话会变得更加烦琐,下面我们来看再增加控制序列会如何。

\begin{texlst}[listing and text]
\newcommand\cmda{abc}
\newcommand\cmdb{def}
\newcommand\cmdc{ghi}
\newcommand\cmdd{jkl}
\let\ea\expandafter

\uppercase\ea\ea\ea\ea\ea\ea\ea{\ea\ea\ea\cmda\ea\cmdb\cmdc}

\uppercase\ea\ea\ea\ea\ea\ea\ea\ea\ea\ea\ea\ea\ea\ea\ea%
{\ea\ea\ea\ea\ea\ea\ea\cmda%
\ea\ea\ea\cmdb%
\ea\cmdc%
\cmdd}
\end{texlst}

在要优先展开4个控制序列的情况下,最外面一层用了15个\texinline{\expandafter},简直太夸张了。不过,可以简单的理解,\texinline{\expandafter}是跳着进行的,所以15个\texinline{\expandafter}命令进行一轮后就剩下7个,第二轮后变成3个,第三轮1个。如果理解了这些的话,你就会明白如果要优先展开5个控制序列的话,最外层需要用31个\texinline{\expandafter}。
